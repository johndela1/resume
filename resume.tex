\documentclass{res}

\begin{document}
\name{John de la Garza}
\address{john@jjdev.com - (714) 895 6311 - Westminster, CA 92683}

\begin{resume}
  \section{SKILLS}
  \begin{itemize}
  \item
    language: Python, Bash, C, Go, Perl, SQL, Lua, Javascript
  \item
      technology: HTTP, RDBMS (PostgreSQL, MySQL), DNS (BIND, Route53), Terraform, Linux (user and kernel), AWS (Lambda, ASG, ECS, ElasticCache (Redis), Kinesis, Security groups, etc)
  \item
      interviewing and mentoring
  \item
      programming
  \end{itemize}

I love programming and building systems. Have worked on things
from mircrocontrollers to distributed systems and everything in
between. Discovering how things work is fun for me and will have no
trouble with new hire learning curve. Prefer to work in a Unix like
environment.

  \section{WORK EXPERIENCE}
  {\large \bf Blackberry, Irvine, CA, senior software engineer \hfill Apr 2019 to Sep 2023}
  \begin{itemize}
  \item
    Was primary contact/maintainer of core system that supported a
    security product.  Worked with small team to do a variety of tasks
    like developing new code, fixing bugs, deployments, sustaining,
    testing, and operations.
  \item
    We where the first team called when there was a customer facing
    issue. We would determine where in the system it was broken, who
    owned it, and most of the time fix it.  This involved tracking down
    the cause in a distributed system with lots of moving parts and
    getting a fix deployed.
  \item
    Frequently pair programmed with others that where newer to the job
    to get them up to speed on using git consistently and learning how
    to run/develop/debug the different micro-services.
  \item
Based on the fact that our core services where IO bound
an asynchronous Python3 framework called 'Tornado' was used for many
of the micro-services. Asyncio in Python is based on coroutines that
cooperatively yield control (instead of blocking) while waiting on
IO/blocking operations. This allows a single process to handle many
requests concurrently.  We had a backend of PostgreSQL (RDBMS) and Redis
(caching and pub/sub).

Other teams did things in Lua and Go. Which we eventually took ownership
of.

  \item
We became experts how things got deployed, on AWS services, and how to
debug when there are many endpoints involved.  A typical week could
looking like working on a P0 production issue, adding new features,
addressing bug reports, doing code review of patches submitted to the
repos we maintained, and tuning (based on performance testing).

  \item
One of the more interesting things was the use of openresty/lua (an app server
that is forked from nginx).  This system had a listening socket on the
Internet that customer devices would connect to with a web-socket, the endpoint
would use this bidirectional socket to send and receive.  Incoming messages
where routed to a set of Kinesis streams where Lambdas would pick them up and
process them.  While outgoing messages originating in a web app (for
example) would get sent to one of  the HTTP micro-services.  This would result
in a series of events that would ultimately publish the commands to a pub/sub
message bus, where the openresty code would pick it up and send it down the socket.
  \end{itemize}

  {\large \bf Comscore, Portland, OR \hfill Nov 2016 to Dec 2018}
  \begin{itemize}

  \item
    Worked on a distributed system which was designed to count ads and
    determine who watched them.  System consisted of retrievers to download
    data from clients, importers to normalize and store data, in house
    parallel database system for the bulk of the data, summarizers to
    analyze the data, RDBMS for storing summary data for
    web front end, queueing subsystem, and exporters to transmit
    results back to clients.

  \item
    Created composable command line tools that acted as a language to
    assist in understanding incoming data and creating custom reports to
    be built combining these tools with some high level scripts.

  \item
    Participated in regular code reviews.

  \item

    Python, Perl, Bash, C, PostgreSQL, Linux

  \end{itemize}

  {\large \bf Idealist.org, Portland, OR \hfill Jul 2015 to Jul 2016}
  \begin{itemize}
    \item
    Worked on a small team creating an HTTPS REST API that needed
    to inter-operate with an existing model, database, and messaging
    system. It provided HTTPS endpoints to allow other systems to post
    jobs in an automated fashion.  This involved doing
    a deep dive into the existing code to fully understand the existing
    architecture.
    \item
    Was involved in the creation of a new system that would allow
    customers to bulk post job listings.  It was available to the end users
    as an REST API where they could submit jobs, check status, etc.
    It was consistent, well documented, and designed to be discoverable.
    \item
    Wrote shell scripts to create chroot based sandbox.  This provided
    a reproducible development environment on individual developer
    computers.
    \item
    There was a lot of collaboration.  Code reviews where almost always
    done.  We frequently pair programmed when it came to work on the
    more tricky parts.  Work was done in two week sprints.
  \item
    Python, Flask, SQLAlchemy, PostgreSQL, Linux, RabbitMQ, Redis, AWS
  \end{itemize}

  {\large \bf Rally, Wash, DC \hfill Jun 2014 to Mar 2015}

  \begin{itemize}

  \item
    Implemented a configuration system which used HTTP to
    internally publish configuration files that lived in a git repository.
    Pushing to the configuration repository triggered the new files
    to be published.  This service was deployed on AWS using Chef.
  \item
    Created a program that would run on a cron job.  It would pull down
    XML data, process, and insert into a local database.
  \item
    Python, Ruby, Linux, shell scripting

  \end{itemize}

  {\large \bf TVplus, Orange, CA \hfill Aug 2012 to Jan 2014}

  \begin{itemize}

  \item
    Designed and implemented a read/write caching layer for with a
    HTTP REST front end.
    Remote data was represented by a graph and the
    local data was a set of trees that could be lazily loaded. 
    and Flask.

  \item
    Designed and implemented a system to daily download XML data
    that represented TV program schedules for the next two weeks and
    merge/load it into a local database.  This could mean scheduled
    shows where removed, added, or rescheduled.  The system would try
    to re-attach content for shows that where rescheduled (if a match
    could be guessed) and flag all other cases of disassociated content
    for human intervention.

  \item
    Created a service that took a URL to a video file and a description
    that specified how to edit the video.  The goal was to mute audio
    and replace video with a black screen for the times and durations
    specified and then transcode to H.264/AAC.

    Once the new video file was created, a hash was generated and uploaded
    to an audio matching service. A request was sent to another server
    to have it generate a different type of hash.  Once this was done
    the path to the second hash was added to a audio matching server.

    Most of this work was done in parallel and utilized multiple cores
    well.

  \item
    Setup development environment using LXC to allow a single server to
    run several systems to approximate our production environment which
    was spread across many amazon EC2 nodes.

  \item
    Python, SQLAlchemy, Flask, PostgreSQL, MongoDB, AWS, ffmpeg

  \end{itemize}

  Formal Education: Cal State Fullerton, BA Management Information
  Systems, 1997, Dean's List

  Languages: English and Spanish

\end{resume}

\end{document}
