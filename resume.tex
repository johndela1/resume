\documentclass{res}
\begin{document}
\name{John de la Garza}
\address{john@jjdev.com - (714) 895 6311 - Westminster, CA 92683}

\pagenumbering{arabic}
\begin{resume}
    Love programming and building systems. Have worked on things
    from microcontrollers to distributed systems and everything in
    between. Discovering how things work and problem solving are two of my
    strengths and will have no trouble with any new hire learning curve.
\section{SKILLS}
\begin{itemize}
  \item
    language: Python, Go, Bash, C, SQL, Perl, Lua, JavaScript, Ruby, Lisp
  \item
    technology: HTTP, RDBMS (PostgreSQL, MySQL), DNS (BIND, Route53),
    Terraform, Linux (user and kernel), AWS (Lambda, ASG, ECS, ElastiCache
    (Redis), Kinesis, Security groups, etc), Nginx, OpenBSD, Tornado,
    Flask
  \end{itemize}
\section{WORK EXPERIENCE}
  {\large \bf Blackberry, Irvine, CA, Senior Software Engineer \hfill Apr 2019 to Sep 2023}
  \begin{itemize}
  \item
    Led design, feature improvements, and maintenance for our core
    systems which provided a REST API backend to a complex enterprise
    security product.  Was responsible for a variety of tasks including
    developing new code, fixing bugs, deployments, sustaining, testing,
    and operations.
  \item
    Was one of the first responders when there was a customer-facing
    issue.  This involved tracking down the cause in a distributed system
    with lots of moving parts and getting a fix deployed.
  \item
    Frequently pair programmed with others that were newer to the job
    to get them up to speed on using git consistently and learning how
    to run/develop/debug the different microservices.
  \item
    Based on the fact that our core services were I/O bound, an
    asynchronous Python3 framework called Tornado was used for many of
    the microservices. Asyncio in Python is based on coroutines that
    cooperatively yield control (instead of blocking) while waiting on
    blocking operations. This allows a single process to handle many
    requests concurrently.  We had a backend of PostgreSQL (RDBMS),
    Redis (caching and pub/sub).
  \item
    Was instrumental in transitioning several services from Python to Go.
  \item
    One of the more interesting things was the use of OpenResty/Lua (an
    application server that is forked from Nginx).  This system had a
    listening socket on the Internet that customer devices would connect
    to with a web-socket, the endpoint would use this bidirectional
    socket to send and receive.  Incoming messages were routed to a set
    of Kinesis streams where Lambdas would pick them up and process them.
    While outgoing messages originating in a web application (for example)
    would get sent to one of  the HTTP microservices.  This would result
    in a series of events that would ultimately publish the commands
    to a pub/sub message bus, where the OpenResty code would pick it up
    and send it down the socket.
  \item
    Async Python, Go, Lua, PostgreSQL, MySQL, AWS, Redis, Kinesis, Lambdas
  \end{itemize}

  {\large \bf Comscore, Portland, OR, Senior Software Engineer \hfill Nov 2016 to Dec 2018}
  \begin{itemize}

  \item
    Worked on a distributed system that was designed to count watched ads
    and determine who watched them.  The system consisted of retrievers to
    download data from clients, importers to normalize and store the bulk
    of the data into a custom in-house parallel database, summarizers
    to store summary data into an RDBMS for use by the web front end,
    queuing subsystem, and exporters to transmit results back to clients.

  \item
    Created composable command line tools that acted as a language to
    assist in understanding incoming data and creating custom reports to
    be built combining these tools with some high level scripts.

  \item
    Participated in regular code reviews.
\item
 \enlargethispage{\baselineskip}
C/C++, Perl, Python, RDBMS

  \end{itemize}

  {\large \bf Idealist.org, Portland, OR, Senior Developer \hfill Jul 2015 to Jul 2016}
  \begin{itemize}
    \item
    Worked on a team creating a REST API that needed
    to inter-operate with an existing model, database, and messaging
    system. It provided HTTPS endpoints to allow other systems to post
    jobs in an automated fashion.  This involved doing
    a deep dive into the existing code to fully understand the existing
    architecture.
    \item
    Was involved in the creation of a new system that would allow
    customers to bulk post job listings.  It was available to the end users
    as a REST API where they could submit jobs, check status, etc.
    It was consistent, well-documented, and designed to be discoverable.
    \item
    Wrote shell scripts to create a chroot based sandbox.  This provided
    a reproducible development environment on individual developer
    computers.
    \item
    There was a lot of collaboration.  Code reviews were almost always
    done.  We frequently pair programmed when working on the
    more tricky parts.  Work was done in two-week sprints.
    \item
    Python, PostgreSQL, AWS
  \end{itemize}

  {\large \bf Rally, Washington, DC, Operations Engineer \hfill Jun 2014 to Mar 2015}

  \begin{itemize}

  \item
    Implemented a configuration system that used HTTP to
    internally publish configuration files that lived in a git repository.
    Pushing to the configuration repository triggered the new files
    to be published.  This service was deployed on AWS using Chef.
  \item
    Created a program that would run on a cron job.  It would pull down
    XML data, process it, and insert it into a local database.
  \item
    Chef, Ruby, Python, AWS
  \end{itemize}

  {\large \bf TVplus, Orange, CA, Senior Developer \hfill Aug 2012 to Jan 2014}

  \begin{itemize}

  \item
    Designed and implemented a read/write caching layer for a
    REST API which was used by a JavaScript front end.
    Remote data was represented by a graph and the
    local data was a set of trees that could be lazily loaded. 

  \item
    Designed and implemented a system to daily download XML data
    that represented TV program schedules for the next two weeks and
    merge/load them into a local database.  This could mean scheduled
    shows were removed, added, or rescheduled.  The system would try
    to re-attach content for shows that were rescheduled (if a match
    could be guessed) and flag all other cases of disassociated content
    for human intervention.

  \item
    Created a service that took a URL to a video file and a description
    that specified how to edit the video.  The goal was to mute audio
    and replace the video with a black screen for the times and durations
    specified and then transcode to H.264/AAC.

    Once the new video file was created, a hash was generated and uploaded
    to an audio-matching service. A request was sent to another server
    to have it generate a different type of hash.  Once this was done
    the path to the second hash was added to an audio-matching server.

    Most of this processing was done in parallel and utilized multiple cores
    well.

  \item
    Set up a development environment using LXC to allow a single server to
    run several systems to approximate our production environment which
    was spread across many Amazon EC2 nodes.
  \item
    Python, Tornado, Flask, PostgreSQL, AWS
  \end{itemize}
Formal Education: Cal State Fullerton, BA Management Information
Systems, Dean's List

Languages: English and Spanish
\end{resume}
\end{document}
