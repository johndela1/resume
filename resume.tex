\documentclass{res}
\usepackage{multicol}
\newsectionwidth{0pt}
\begin{document}
\name{John de la Garza}
\address{Westminster, CA  92683 \\ email: john@jjdev.com \\ phone: 657 400 9422}

\begin{resume}
\thispagestyle{empty}
\section{\centerline{Summary of Qualifications}}
John de la Garza has been programming in a Linux environment for over
10 years and has programmed in a variety of languages.
He has expertise in backend/systems/network programming, working with
databases, sysadmin tasks (DNS, web servers, firewall).  He is constantly
learning and always looking to automate, minimize complexity, and
create systems that are stable and easy to understand and debug when
issues arise.

\begin{itemize}
\item
Languages:  C, Python, Bash, Lua, Scheme, \LaTeX, IA32 assembly
\item
Systems:  Linux (x86, ARM, PPC), OpenBSD, Solaris (x86) U-Boot, KVM,
AT91Bootstrap
\item
Education: Cal State Fullerton, BA Management Information Systems, 1997, Dean's List
\end{itemize}

\section{\centerline{EXPERIENCE}}
{\large \bf TVplus, Orange, CA \hfill August 2012 to Jan 2014}
\begin{itemize}
\item Designed, documented, and wrote a set of programs that would fetch
data from third party sources  and transform them
for local use.  This involved meeting with management and figuring
out how the data where going to be presented.  Once decided upon,
the data structures were created, some ideas were prototyped, and
the program was written.
The program transformed the raw data to a data structure that
was smaller and just enough for the presentation.  This new structure
was stored in the local database for direct use by the presentation
layer.  The presentation was accessed from a REST api that provided
the data using JSON.  The presentation code would do some minor
transforms (flattening or showing a specific node in the tree) on the
stored data and present the results.  This was done with python, and
SQLAlchemy (on PostgreSQL).  The front end was done with Flask.

\item Rewrote a program that was used to download schedule and program
data in the form of XML.  The data were added to and merged with existing
data on a nightly basis.  This program ran in the background and
proved to be very stable.  This was written in Python and used
MongoDB as a database.

\item Automated video processing work.  Before this work, there
was a person who had to take videos and cut them up, insert blanks spots,
and mute commercials, based on a set of time offsets provided to them.  They
also had to generate a hash and upload it to a third part and then edit
a remote JSON config file.  This was automated using Python and ffmpeg.
The user would submit the offset file and the program would do the rest.

\item Responsible for the sysadmin tasks for our in house
network and servers.  Dealt with routing, firewall, working with ISP,
DNS, installing software, etc.  Ran several database servers, a git
server over ssh, iptables, BIND, Nginx.  

\item Participated in peer reviews of code.

\item Linux, Python, SQL Expression Language (SQLAlchemy), PostgreSQL, MongoDB
\end{itemize}

{\large \bf Cisco, Irvine, CA (contractor) \hfill Oct 2011 to Jan 2012}
\begin{itemize}
\item Fixed bugs in configuration firmware that was used to configure
 routing tables, TCP/IP, and wireless settings.
\item Cross compiled Linux for ARM and MIPS architectures
\item Modified SAMBA to work better with OS X.
\end{itemize}

{\large \bf Axcelion, Huntington Beach, CA (contractor) \hfill Jan 2011 to Oct 2011}
\begin{itemize}
\item Added  C code to existing scheduler to allow cluster statistics to
be accessed via TCP/IP socket.
\item Created Python layer to interface JavaScript AJAX calls with
socket data via JSON/RPC.
\item Wrote caching layer to speed up the web interface at the cost of
less timely data.
\item Built a JavaScript interface to view statistics of a cluster with
asynchronous updates, we used a tree data structure that loaded on
demand as one navigated the tree.
\item Created the front end in JavaScript (Qooxdoo framework)
\item Wrote and documented Python module to allow Python code to
access cluster statistics
\item Linux, C, Python, JavaScript, Qooxdoo
\end{itemize}

{\large \bf RK Hydroponics, Huntington Beach, CA \hfill Mar 2009 to Nov 2010}
\begin{itemize}
\item Got U-Boot and mainline kernel running on AT91 ARM hardware.
\item Set up a basic system with custom kernel and Busy-Box.
\item Created control software that was used in controlling salt levels
in a hydroponic nutrient solution.  Samples would be read in from
EC (electrical conductivity) sensor and used as feedback to a control
loop which would add or withhold water in order to keep salinity from rising.
This prevented large swings in the salinity level, which made it possible
to reuse the solution several times as opposed to being dumped after a 
single use.
\end{itemize}

{\large \bf Teradek, Irvine, CA (contractor) \hfill Jul 2008 to Dec 2008}
\begin{itemize}
\item Developed software in C that would run on ARM and PowerPC
\item Worked with TCP, UDP, and Unix sockets to manage video streams for
a device with multiple cameras as inputs.
\item Created Lua scripts to control video codec.
\item Cross compiled kernel modules and user space programs for various
pieces of hardware on an embedded Linux platform; Devices such as USB
ports, GPS modules, 802.11g adapters
\end{itemize}

{\large \bf Mojix, Los Angeles, CA (contractor) \hfill Jan 2008 to Apr 2008}
\begin{itemize}

\item Created an application that allowed an end user to interact
with lower level motion tracking system.  Users could specify polygons
that would tell the system where to look for motion.  This involved
building a GUI.  Some initial work was done with Tcl/Tk, but eventually
we decided to write it in Python using PyGtk.

\item Worked on motion detection system that used a CCD camera to sense
events then send a message over the network (using UDP) to another
machine that would attempt to read a RFID tag in the area.

\end{itemize}

{\large \bf Tadpole, Torrance, CA (contractor) \hfill Aug 2007 to Dec 2007}
\begin{itemize}

\item Managed Apache web server that was used for running bug
tracking system and web mail.

\item Built custom Linux distribution to be installed on in-car
computer where our navigation system was going to run.

\item Wrote Python program that would let users view their location
map, zoom, and scroll back in time to see where they had come from.


\end{itemize}

{\large \bf CreditVision, Pasadena, CA \hfill Nov 2006 to Jan Aug 2007}
\begin{itemize}
\item
Worked on existing web application that used mod\_perl, Apache, and Oracle to
provide customers with merged credit reports.  When he started this
job, the previous developer had just left on short notice.  Things where
not stable and there was no documentation.  He got up to speed, fixed some
of the bugs that caused the stability issues, and helped expand the system
to provide redundancy for fail over.  When he left the company they had
a system that was well documented and required very little maintenance.

\item
Installed and administered Linux, Apache, Oracle, Open BSD on off
site server located in a nearby data center.

\end{itemize}

{\large \bf ETAS, Santa Barbara, CA (contractor) \hfill Jan 20 to Jun 2006}
\begin{itemize}
\item Maintained and tested Linux kernel to run on an embedded
system (ARM) for use in automotive industry.
\item Involved forward porting of existing patches, and locating and applying
new patches.
\end{itemize}


{\large \bf First American RES, Anaheim, CA \hfill May 2001 to April 2005}
\begin{itemize}
\item Wrote and maintained web-apps written mainly in Java that ran
with Apache, Tomcat, and Weblogic on Linux and Solaris.
\item Designed and debugged automate build process.
\item Maintained 40 to 45 production Linux servers that serve as
web-servers, proxy servers, and application servers.
\item Created and maintained configuration files,
SSL keys and certificates, shell scripts used to deploy apps and
analyze log data to create reports.
\item Developed remote deployment system using shell scripting and
ssh.
\item  Worked with Solaris, Oracle, Weblogic, Apache, shell, Perl, Python,
SQL, Linux, AIX, C, Java
\end{itemize}


Bilingual in English and Spanish

References available upon request

\end{resume}
\end{document}


