\documentclass{res}
\usepackage{multicol}
\usepackage{fancyhdr}
\pagestyle{fancy}
\fancyhf{}
\makeatletter
\fancyhead[L]{\textbf{John de la Garza - john@jjdev.com}}
\makeatother
\fancyhead[R]{\thepage}
\fancyheadoffset[L]{\sectionwidth}
\setlength\headheight{12pt}
\setlength\headsep{3pt}
\addtolength\topmargin{-15pt}
\AtBeginDocument{\thispagestyle{empty}}

\newsectionwidth{0pt}
\begin{document}
\name{John de la Garza}
\address{Westminster, CA  92683 \\ email: john@jjdev.com \\ phone: 657 400 9422}

\begin{resume}
\thispagestyle{empty}

\section{\centerline{Summary of Qualifications}}

\begin{itemize}

\item Programming:
	C, Python, Bash, Awk, Ruby, Lisp, Lua, Perl, Tcl/TK, \LaTeX, x86
	and ARM assembler, SQL

\item Expertise:
XXX  opensource/free software
booting linux
	network and systems programming, back end systems, embedded
	systems, database driven applications, TCP/IP, kernel,
	GNU toolchain (GCC, make, GDB, Binutils), git

	Linux (x86, ARM), OpenBSD, U-Boot

\item Education:
	Cal State Fullerton, BA Management Information Systems, 1997,
	Dean's List

\end{itemize}

\section{\centerline{Experience}}

{\large \bf Rally, Wash, DC \hfill Jun 2014 to Mar 2015}

\begin{itemize}

\item
	did stuff
\end{itemize}

{\large \bf TVplus, Orange, CA \hfill August 2012 to Jan 2014}

\begin{itemize}

\item
	Implemented cache layer for a large remote database.  Remote
	data was represented by a graph and the local data was a
	set of trees.  Each tree represented an entity or topic in
	the graph.  The cache provided read and write operations.  Write
	operations made sure local changes where available immediately and
	pushed back to the remote database.  There was a configuration
	system for specifying how remote structures would be mapped
	to local structures based on their type.  The top layer was an
	interface that provided functions to query and edit the data.

\item
	Wrote a program for reconciling schedule changes in a system
	that managed TV show schedules and associated content.	A database
	contained the schedule for next two weeks and associated content
	was added to scheduled events based on their network (CBS, FOX,
	etc) and their start time.  Every 24 hours the schedules would
	get updated.  This could mean events where removed, added,
	or moved.  My system would try to reattach content for shows
	that moved in time (if a match could be guessed) and flag all
	other cases of disassociated content for human intervention.
\item
	Created a system for transforming and hashing videos.  The
	input to this was a path to a video file and a list of tuples
	containing an operation code, time stamp, and a duration value.
	The two operations where to either only mute audio or to mute
	audio and replace video with a black screen for the time and
	duration specified.  When requests came in they where validated
	and added to a persistent queue.  To process an item, a video
	file would be downloaded, then transcoded to H.264/AAC.  The file
	would be cut into pieces based on time and duration inputs then
	reassembled with muted sections and/or blacked out sections.
	Lots of this work was done in parallel and utilized 16 cores well.
	Once this was done a hash was generated on the audio file and
	uploaded to our audio matching server.	A request was sent to
	another server to have it generate a different type of hash.
	Once this was done the path to the second hash was added to the
	audio matching server.

\item
	Setup  development  environment using LXC to allow a single
	server to run several systems to approximate our production
	environment which was spread across many amazon EC2 nodes.

\item
	Dealt with routing, firewall, working with ISP, DNS, general
	systems administration.  Used ffmpeg, PostgreSQL, MongoDB,
	git server over ssh, iptables, BIND, Apache, Nginx, Python,
	Flask, gevent.

\end{itemize}

{\large \bf Cisco, Irvine, CA (contractor) \hfill Oct 2011 to Jan 2012}

\begin{itemize}

\item
	Fixed bugs in configuration firmware that was used to configure
	TCP/IP settings on residential Cisco routers.
\item
	Cross compiled Linux kernel and U-boot for ARM and MIPS
	architectures
\item
	I left the position due to cancellation of product I was
	working on.  The person I worked under can provide a
	reference.

\end{itemize}

{\large \bf RK Hydroponics, Huntington Beach, CA \hfill Mar 2009 to
Oct 2011}

\begin{itemize}

\item
	Brought up U-Boot and the mainline kernel on AT91 ARM hardware.
	Created a simple Linux system with BusyBox and small custom
	Linux kernel.

\item
	Created control and logging software that was used in controlling
	salt levels in a hydroponic nutrient solution and ran on
	small ARM board.  Samples would be read in from EC (electrical
	conductivity) sensor and used as feedback to a control loop which
	would add or withhold water in order to keep salinity from rising.
	This prevented large swings in the salinity level, which made
	it possible to reuse the solution several times as opposed to
	being dumped after a single use.  There was one controller per
	reservoir.  Data about water usage and EC readings where sent
	to a central database.

\item
	Front end allowed reporting based on statistical data and allowed
	for configuration information to be passed to controller units.

\end{itemize}

{\large \bf Tadpole, Torrance, CA \hfill Aug 2007 to Dec 2008}

\begin{itemize}

\item
	Worked with TCP, UDP, and Unix sockets to manage video streams
	for a device with multiple cameras as inputs.

\item
	Created Lua scripts to control hardware video codec.

\item
	Cross compiled kernel modules and user space programs for various
	pieces of hardware on an embedded Linux platform; devices such
	as USB controllers, GPS modules, 802.11g adapters.

\item
	Developed software in C that would run on ARM and PowerPC platforms.

\end{itemize}

{\large \bf CreditVision, Pasadena, CA \hfill Nov 2006 to Aug 2007}

\begin{itemize}

\item
	Worked on existing web application that used mod\_perl, Apache,
	and Oracle to provide customers with merged credit reports.
	When I started this job, the previous developer had just
	left on short notice.  Things where not stable and there was
	no documentation.  I got up to speed, fixed some of the bugs
	that caused the stability issues, and helped expand the system
	to provide redundancy for fail over.  When he left the company
	they had a system that was well documented and required very
	little maintenance.

\item
	Installed and administered Linux, Apache, Oracle, Open BSD on
	off site servers located in a nearby data center.
\end{itemize}

{\large \bf ETAS, Santa Barbara, CA (contractor) \hfill Jan 2006 to
Jun 2006}

\begin{itemize}

\item
	Maintained and tested Linux kernel to run on an embedded system
	(ARM) for use in automotive industry.

\item
	Involved forward porting of existing patches, and locating and
	applying new patches.

\end{itemize}


{\large \bf First American RES, Anaheim, CA \hfill May 2001 to April 2005}

\begin{itemize}

\item
	Wrote and maintained web-apps written mainly in Java that ran
	with Apache, Tomcat, and Weblogic on Linux and Solaris.
\item
	Designed and debugged automate build process.
\item
	Maintained 40 to 45 production Linux servers that serve as
	web-servers, proxy servers, and application servers.
\item
	Created and maintained configuration files, SSL keys and
	certificates, shell scripts used to deploy apps and analyze log
	data to create reports.

\item
	Developed remote deployment system using shell scripting and ssh.

\item
	Worked with Solaris, Oracle, Weblogic, Apache, bash, Perl,
	Python, SQL, Linux, AIX, C, Java.

\end{itemize}

Bilingual in English and Spanish

References available upon request

\end{resume}

\end{document}
