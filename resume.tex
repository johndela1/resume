\documentclass{res}
\usepackage{multicol}
\usepackage{fancyhdr}
\pagestyle{fancy}
\fancyhf{}
\makeatletter
\fancyhead[L]{\textbf{John de la Garza - john@jjdev.com}}
\makeatother
\fancyhead[R]{\thepage}
\fancyheadoffset[L]{\sectionwidth}
\setlength\headheight{12pt}
\setlength\headsep{3pt}
\addtolength\topmargin{-15pt}
\AtBeginDocument{\thispagestyle{empty}}

\newsectionwidth{0pt}
\begin{document}
\name{John de la Garza}
\address{Portland, OR 97232 \\ email: john@jjdev.com \\ phone: 657 400 9422}

\begin{resume}
  \thispagestyle{empty}

  \section{\centerline{Expertise}}

  I have worked at many levels of software stacks on things like network
  programming, embedded systems, database driven applications.  I familiar
  with tweaking and running Linux on various platforms.

  C, Python, Bash, Awk, Ruby, Lisp, Lua, Perl, Tcl/TK, \LaTeX, x86 and
  ARM assembler, SQL

  git, GNU toolchain (GCC, make, GDB, binutils), TCP/IP, KVM,
  U-Boot, flask HTTP and REST frameworks, PostgreSQL

  Formal Education:
  Cal State Fullerton, BA Management Information Systems, 1997,
  Dean's List

  \section{\centerline{Experience}}
          {\large \bf Action without boarders \hfill Jul 2015 to July 2016}
          \begin{itemize}

          \item
            I started on a team that was in the process of creating a
            REST API that was reachable over HTTP.  The API provided
            endpoints which allowed single page javascript apps to do
            CRUD operations on the same backend that the traditional
            multipage web was using.
            
          \item
            Worked on a team which was tasked with creating a new
            system that needed to closely interact with the internals
            of an existing application.  We worked closely with
            python, postgresql, flask, sqlalchemy, and git.

            This required learning about an existing model, which was
            implemented using a custom fork of sqlalchemy and
            postgresql.  After learning the existing system we built
            our system to interoperate with the the existing database
            and event system.

            We also created a small test suite which ran a limited
            number of test iteratively over several data sets which
            represented, both sides of all the possible edge cases.
            We where able to deliver working code with a bounded set
            of tests.

          \end{itemize}

          {\large \bf Rally, Wash, DC \hfill Jun 2014 to Mar 2015}

          \begin{itemize}

          \item
	    Worked on team of 5 people to build tools around a
            cloud-based environment.  We used some custom python code
            to manage different cloud environments and we used Chef to
            deploy code across 100's of AWS nodes.

          \item
	    Wrote a set of programs in Ruby which were used in a large
            port from one type of configuration system (attributes and
            chef environment) to a existing in house system that was
            newly created to make the data available over HTTPS.

          \item

	    Re-purposed many of the programs described above to be
            used in a library which was used to make common changes
            across many configuration databases as we went forward
            with the new configuration system.

          \item

    	    Participated in 7x24 on-call rotation.

          \end{itemize}

          {\large \bf TVplus, Orange, CA \hfill August 2012 to Jan 2014}

          \begin{itemize}

          \item
	    Implemented cache layer for a large remote database.
            Remote data was represented by a graph and the local data
            was a set of trees.  Each tree represented an entity or
            topic in the graph.  The cache provided read and write
            operations.  Write operations made sure local changes
            where available immediately and pushed back to the remote
            database.  There was a configuration system for specifying
            how remote structures would be mapped to local structures
            based on their type.  The top layer was an interface that
            provided functions to query and edit the data.

          \item
	    Wrote a program for reconciling schedule changes in a
            system that managed TV show schedules and associated
            content. A database contained the schedule for next two
            weeks and associated content was added to scheduled events
            based on their network (CBS, FOX, etc) and their start
            time.  Every 24 hours the schedules would get updated.
            This could mean events where removed, added, or moved.  My
            system would try to re-attach content for shows that moved
            in time (if a match could be guessed) and flag all other
            cases of disassociated content for human intervention.
          \item
	    Created a system for transforming and hashing videos.  The
            input to this was a path to a video file and a list of
            tuples containing an operation code, time stamp, and a
            duration value.  The two operations where to either only
            mute audio or to mute audio and replace video with a black
            screen for the time and duration specified.  When requests
            came in they where validated and added to a persistent
            queue.  To process an item, a video file would be
            downloaded, then transcoded to H.264/AAC.  The file would
            be cut into pieces based on time and duration inputs then
            reassembled with muted sections and/or blacked out
            sections.  Lots of this work was done in parallel and
            utilized 16 cores well.  Once this was done a hash was
            generated on the audio file and uploaded to our audio
            matching server. A request was sent to another server to
            have it generate a different type of hash.  Once this was
            done the path to the second hash was added to the audio
            matching server.

          \item
	    Setup development environment using LXC to allow a single
            server to run several systems to approximate our
            production environment which was spread across many amazon
            EC2 nodes.

          \item
	    Dealt with routing, firewall, working with ISP, DNS,
            general systems administration.  Used ffmpeg, PostgreSQL,
            MongoDB, git server over ssh, iptables, BIND, Apache,
            Nginx, Python, Flask, gevent.

          \end{itemize}

          {\large \bf Cisco, Irvine, CA (contractor) \hfill Oct 2011 to Jan 2012}

          \begin{itemize}

          \item
	    Worked on a small team to do bug fixes for existing
            firmware that was used to configure network settings on
            routers over a HTTP interface.  This was mostly done in C.
            These devices used U-Boot and ran Linux on ARM and MIPS
            chips.

          \item
	    Wrote various shell scripts that where run at boot time and did
	    basic setup for a media sharing service using Samba.

          \item
	    I left the position due to cancellation of product I was
	    working on.  The person I worked under can provide a reference.
          \item

          \end{itemize}

          {\large \bf RK Hydroponics, Huntington Beach, CA \hfill Mar 2009 to
            Oct 2011}

          \begin{itemize}

          \item
	    Brought up U-Boot and the mainline Linux kernel on AT91 ARM
	    hardware.  Created a simple Linux system with BusyBox and small
	    Linux kernel.

          \item
	    Created firmware and device driver for an ATmega328P MCU
	    based electrical conductivity (EC) meter which provided an I2C
	    interface to a circuit that would measure the EC of a solution.
	    The EC value was used to estimate the salinity of a solution.
	    The firmware provided a way to read, calibrate, get statistics,
	    and persists settings to an EEPROM.

          \item
	    Wrote user space control and logging software that was used in
	    controlling salt levels in a hydroponic nutrient solution and
	    ran on the ARM board.  EC readings, from the device described
	    above, where used as feedback to a control loop which would add
	    or withhold water in order to keep salinity levels stable.
	    This prevented large swings in the salinity level, which made
	    it possible to reuse the solution several times as opposed to
	    being dumped after a single use.  There was one controller per
	    reservoir.  Data about water usage and EC readings where sent
	    to a central database over UDP using wifi.


          \end{itemize}

          {\large \bf Tadpole, Torrance, CA \hfill Aug 2007 to Dec 2008}

          \begin{itemize}

          \item
	    Wrote device drivers for various pieces of custom hardware such
	    as clocks, input devices, hardware video codecs.

          \item
	    Worked with TCP, UDP, and Unix sockets to manage video streams
	    for a device with multiple cameras as inputs.

          \item
	    Created Lua scripts to control hardware video codec.

          \item
	    Cross compiled kernel modules and user space programs for various
	    pieces of hardware on an embedded Linux (ARM and PPC) platform;
	    devices such as USB controllers, GPS modules, 802.11g adapters.

          \end{itemize}

          {\large \bf CreditVision, Pasadena, CA \hfill Nov 2006 to Aug 2007}

          \begin{itemize}

          \item
	    Worked on existing web application that used mod\_perl, Apache,
	    and Oracle to provide customers with merged credit reports.
	    When I started this job, the previous developer had just
	    left on short notice.  Things where not stable and there was
	    no documentation.  I got up to speed, fixed some of the bugs
	    that caused the stability issues, and helped expand the system
	    to provide redundancy for fail over.  When I left the company
	    they had a system that was well documented and required very
	    little maintenance.

          \item
	    Installed and administered Linux, Apache, Oracle, Open BSD on
	    off site servers located in a nearby data center.
          \end{itemize}

          {\large \bf ETAS, Santa Barbara, CA (contractor) \hfill Jan 2006 to
            Jun 2006}

          \begin{itemize}

          \item
	    Maintained and tested Linux kernel to run on an embedded system
	    (ARM) for use in automotive industry.

          \item
	    Involved forward porting of existing patches, and locating and
	    applying new patches.

          \end{itemize}


          {\large \bf First American RES, Anaheim, CA \hfill May 2001 to April 2005}

          \begin{itemize}

          \item
	    Wrote and maintained web-apps written mainly in Java that ran
	    with Apache, Tomcat, and Weblogic on Linux and Solaris.
          \item
	    Designed and debugged automate build process.
          \item
	    Maintained 40 to 45 production Linux servers that serve as
	    web-servers, proxy servers, and application servers.
          \item
	    Created and maintained configuration files, SSL keys and
	    certificates, shell scripts used to deploy apps and analyze log
	    data to create reports.

          \item
	    Developed remote deployment system using shell scripting and ssh.

          \item
	    Worked with Solaris, Oracle, Weblogic, Apache, bash, Perl,
	    Python, SQL, Linux, AIX, C, Java.

          \end{itemize}

          Bilingual in English and Spanish

          References available upon request

\end{resume}

\end{document}
