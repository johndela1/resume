\documentclass{res}

\begin{document}

\name{John de la Garza}

\address{john@jjdev.com - (657) 400 9422 - Westminster, CA 92683}

\begin{resume}
  \section{SKILLS}

  \begin{itemize}

  \item
    language: Python, C, Lua, Perl, SQL, shell scripting, Javascript, Go
  \item
    technologies: Linux, PostgreSQL, Tornado, Flask, Django, MySQL, Redis, AWS
  \end{itemize}

  \section{WORK EXPERIENCE}

  {\large \bf Blackberry/Cylance, Irvine, CA \hfill April 2019 to Sep 2023}
  \begin{itemize}
  \item
  did random stuff
  \end{itemize}

  {\large \bf Comscore, Portland, OR \hfill Nov 2016 to Dec 2018}
  \begin{itemize}

  \item
    Worked on a distributed system which was designed to count ads and
    determine who watched them.  System consisted of retrievers to download
    data from clients, importers to normalize and store data, in house
    parallel database system for the bulk of the data, summarizers to
    analyze the data, RDBMS for storing summary data for
    web front end, queueing subsystem, and exporters to transmit
    results back to clients.

  \item
    Created composable command line tools that acted as a language to
    assist in understanding incomming data and creating custom reports to
    be built combining these tools with some high level scripts.

  \item
    Participated in regular code reviews.

  \item

    Python, Perl, Bash, C, PostgreSQL, Linux

  \end{itemize}

  {\large \bf Idealist.org, Portland, OR \hfill Jul 2015 to Jul 2016}

  \begin{itemize} \item
    Worked on a small team creating an HTTPS REST API that needed
    to inter-operate with an existing model, database, and messaging
    system. It provided HTTPS endpoints to allow other systems to post
    jobs in an automated fashion.  This involved doing
    a deep dive into the existing code to fully understand the existing
    architecture.

    \item

    Was involved in the creation of a new system that would allow
    customers to bulk post job listings.  It was available to the end users
    as an REST API where they could submit jobs, check status, etc.
    It was consistent, well documented, and designed to be discoverable.

  \item

    Wrote shell scripts to create chroot based sandbox.  This provided
    a reproducible development environment on individual developer
    computers.

    \item

    There was a lot of collaboration.  Code reviews where almost always
    done.  We frequently pair programmed when it came to work on the
    more tricky parts.  Work was done in two week sprints.

  \item

    Python, Flask, SQLAlchemy, PostgreSQL, Linux, RabbitMQ, Redis, AWS

  \end{itemize}

  {\large \bf Rally, Wash, DC \hfill Jun 2014 to Mar 2015}

  \begin{itemize}

  \item
    Implemented a configuration system which used HTTP to
    internally publish configuration files that lived in a git repository.
    Pushing to the configuration repository triggered the new files
    to be published.  This service was deployed on AWS using Chef.
  \item
    Created a program that would run on a cron job.  It would pull down
    XML data, process, and idempotently insert into a local database.
  \item
    Python, Ruby, Linux, shell scripting

  \end{itemize}

  {\large \bf TVplus, Orange, CA \hfill Aug 2012 to Jan 2014}

  \begin{itemize}

  \item
    Designed and implemented a read/write caching layer for with a
    HTTP REST front end.
    Remote data was represented by a graph and the
    local data was a set of trees that could be lazily loaded. 
    and Flask.

  \item
    Designed and implemented a system to daily download XML data
    that represented TV program schedules for the next two weeks and
    merge/load it into a local database.  This could mean scheduled
    shows where removed, added, or rescheduled.  The system would try
    to re-attach content for shows that where rescheduled (if a match
    could be guessed) and flag all other cases of disassociated content
    for human intervention.

  \item
    Created a service that took a URL to a video file and a description
    that specified how to edit the video.  The goal was to mute audio
    and replace video with a black screen for the times and durations
    specified and then transcode to H.264/AAC.

    Once the new video file was created, a hash was generated and uploaded
    to an audio matching service. A request was sent to another server
    to have it generate a different type of hash.  Once this was done
    the path to the second hash was added to a audio matching server.

    Most of this work was done in parallel and utilized multiple cores
    well.

  \item
    Setup development environment using LXC to allow a single server to
    run several systems to approximate our production environment which
    was spread across many amazon EC2 nodes.

  \item
    Python, SQLAlchemy, Flask, PostgreSQL, MongoDB, AWS, ffmpeg

  \end{itemize}

  {\large \bf Cisco, Irvine, CA (contractor) \hfill Oct 2011 to Jan 2012}

  \begin{itemize}

  \item
    Worked on a team assigned to fix existing bugs on a legacy
    application.  I worked closely with the QA team.  The application was
    a tool that ran on residential Cisco router.  It provided a web page
    interface that allowed users to configure the network settings. I
    worked at many layers.  Sometimes fixing front end Javascript issues
    other times I was looking at system level C code, and shell scripts.
    This involved cross compiling for the ARM and MIPS architectures.

  \item
    Wrote various shell scripts that where run at boot time and did
    basic setup for a media sharing service using Samba.

  \item
    I was working as a contractor.  My contract was not extended due to
    cancellation of the product.  The person I worked
    under can provide a reference.

  \end{itemize}


  Formal Education: Cal State Fullerton, BA Management Information
  Systems, 1997, Dean's List

  Languages: English and Spanish

\end{resume}

\end{document}
